% 
\documentclass[10pt]{article}
\usepackage{amscd,amsfonts,amssymb,amstext,latexsym} 
\usepackage{amsmath,mathbbol,mathrsfs,stmaryrd, mathtools} 
%\usepackage{mathbbol,mathrsfs,stmaryrd}
\usepackage {algorithm2e} 
\usepackage{theoremref}
\usepackage[T1]{fontenc}
\usepackage[english]{babel} 
\usepackage {enumerate}
\usepackage{url}
%\usepackage {algpseudocode}  
\usepackage{graphics} 
\usepackage{tikz}
%\usepackage[square]{natbib}
\usepackage[width=14.8cm,left=3cm]{geometry}
\usetikzlibrary{automata,calc}
%\usepackage{tgtermes} 
\usepackage{listings}
\usepackage{mathptmx}
\usepackage{fancyhdr}
\usepackage{verbatim}
\usepackage{float}
\usepackage{enumitem}
\usepackage{booktabs}
\usepackage[flushleft]{threeparttable}
\usepackage{listings}
\usepackage{verbatim}
\usepackage{fancyhdr}
\usepackage{multirow,multicol}
\usepackage[colorlinks=true,linkcolor=blue,citecolor=blue,urlcolor=blue]{hyperref}
\usepackage{tabto}
\lstset{ %
language=C++,                % choose the language of the code
basicstyle={\ttfamily},       % the size of the fonts that are used for the code
backgroundcolor=\color{white},  % choose the background color. You must add \usepackage{color}
showspaces=false,               % show spaces adding particular underscores
aboveskip=6mm, 
%belowskip=3mm, 
numbers=left, numberfirstline=false, numberblanklines=false,
numberstyle=\tiny\color{gray}, numbersep= 5pt, 
showstringspaces=false,         % underline spaces within strings
showtabs=false,                 % show tabs within strings adding particular underscores
%frame=single,           % adds a frame around the code
%frame = tb, 
frame = none, 
tabsize=2,          % sets default tabsize to 2 spaces
captionpos=b,           % sets the caption-position to bottom
breaklines=true,        % sets automatic line breaking
breakatwhitespace=false,    % sets if automatic breaks should only happen at whitespace
escapeinside={\%*}{*)}          % if you want to add a comment within your code
}
%\graphicspath{{../../pics/}}
\fancypagestyle{plain}{
\fancyhf{}
\rhead{School of Computer Science and Applied Mathematics\\ 
%\noindent\rule{15.4cm}{0.4pt}\\
\footnotesize{\textsc{University of the Witwatersrand, Johannesburg}}}
\lhead{\includegraphics[scale=0.08]{witslogo_h.png}}
\fancyfoot[C]{\thepage}
\renewcommand{\headrulewidth}{0.4pt}
}

\textwidth=16.8cm 
\textheight=22.6cm 
\evensidemargin 0pt 
\oddsidemargin 0pt 
\leftmargin 0pt 
\rightmargin 0pt 
\setlength{\topmargin}{0pt} 
\setlength{\footskip}{50pt}
\setlength{\parindent}{0pt}
\setlength{\parskip}{1em}
\linespread{1} 
% 
\makeatletter
\newcommand{\rmnum}[1]{\romannumeral #1}
\newcommand{\Rmnum}[1]{\expandafter\@slowromancap\romannumeral #1@}
\makeatother

\begin{document}
\title{COMS4036A Project - Gaussian Mixture Models}
\author{Angus Mackenzie}
\date{\today} 
\maketitle 
%\thispagestyle{empty}
\pagestyle{fancy}
\fancyhf{}
\fancyhead[R]{\thepage}
\fancyhead[L]{COMS4036A}
%\vskip 3mm 
%\pagenumbering{roman}
%\newpage
\pagenumbering{arabic}
\section{Introduction}
In this project we were tasked with building a Gaussian Mixture Model (GMM) \cite{princeCVMLI2012} that can detect coins in an image. The aim was to produce a mask that would show coin pixels in white and desk pixels in black. 
Computer Vision \cite{princeCVMLI2012}, Approach, Technical Details: Covariance, Colour Space, 1-10 Components, Accuracy. Investigate resulting component distributions and provide analysis on their position and co-variances: do the components align with different types of coins? 

\section{Experimental Methodology}
\subsection{Approach}
The fundamental algorithm of a GMM is Expectation Maximisation \cite{princeCVMLI2012}, which consist of a expectation step (E-step) and a Maximisation step (M-step). Initially, the two algorithms for both steps were completed in order to test the accuracy of their output on dummy data. 

Then a method for reading in images and masks for training was created. Once those were in place a simple bit of research was one in order to ascertain which colour space would be modelled easiest. 

After that the images were run through the algorithm and the output and accuracy of the output ascertained.

\section{Results \& Analysis}

\section{Conclusion}


%\bibliographystyle{apalike} I prefer plain references, but I will ask Hairong if that is okay
\bibliographystyle{plain}
\bibliography{references}
\end{document} 
